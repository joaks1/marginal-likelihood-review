By providing a framework of accounting for the shared ancestry inherent to all
life,
phylogenetics is becoming the statistical foundation of biology.
The importance of model choice continues to grow as phylogenetic models
continue to increase in complexity to better capture micro and
macroevolutionary processes.
% A Bayesian approach to such rich models is often necessary to make them
% tractible.
In a Bayesian framework, 
the marginal likelihood is how data update our prior beliefs about models,
which gives us an intuitive measure of comparing model fit that is grounded in
probability theory.
Given the rapid increase in the number and complexity of phylogenetic models,
methods for approximating marginal likelihoods are increasingly important.
Here we try to provide an intuitive description of marginal likelihoods and why
they are important in Bayesian model testing.
We also categorize and review methods for estimating marginal likelihoods of
phylogenetic models.
In doing so, we use simulations to evaluate the performance of one such method
based on approximate-Bayesian computation (ABC) and find that \vmadd{it} is biased as
predicted \vmdel{from} \vmadd{by} theory.
Furthermore, we review some applications of marginal likelihoods to
phylogenetics, highlighting how they can be used to learn about models of
evolution from biological data.
We conclude by discussing the challenges of Bayesian model choice and future
directions that promise to improve the approximation of marginal likelihoods
and Bayesian phylogenetics as a whole.
