\mFigure{images/bayes-demo.pdf}{
    An illustration of the posterior probability densitites and marginal
    likelihoods of the four different prior assumptions we made in our
    coin-flipping experiment.
    The data are 50 ``heads'' out of 100 coin flips, and the parameter,
    \probheads, is the probability of the coin landing heads side up.
    The binomial likelihood density function is proportional to a
    $\textrm{Beta}(51, 51)$ and is the same across the four different beta
    priors on \probheads ($M_1$--$M_4$).
    The posterior of each model is a $\textrm{Beta}(\alpha + 50, \beta + 50)$
    distribution.
    The marginal likelihoods ($P(D)$; the average of the likelihood density
    curve weighted by the prior) of the four models are compared.
}{fig:bayesDemo}

\mFigure{images/abc-glm-bayes-factors.pdf}{
    A comparison of the approximate-likelihood Baysian computation general
    linear model (ABC-GLM) estimator of the marginal likelihood
    \citep{Leuenberger2010} to quadrature integration approximations
    \citep{Xie2011} for 100 simulated datasets.
    We compared the ratio of the marginal likelihood (Bayes factor) comparing
    the correct branch-length model
    [branch length $\sim$ uniform(0.0001, 0.1)]
    to a model with a boader prior on the branch length
    [branch length $\sim$ uniform(0.0001, 0.2)].
    The solid line represents perfect performance of the ABC-GLM estimator
    (i.e., matching the ``true'' value of the Bayes factor).
    The dashed line represents the expected Bayes factor when failing to
    penalize for the extra parameter space (branch length 0.1 to 0.2) with
    essentially zero likelihood.
    Quadrature integration with 1,000 and 10,000 steps using the rectangular
    and trapezoidal rule produced identical values of log marginal likelihoods
    to at least five decimal places for all 100 simulated datasets.
}{fig:glmPerformance}
